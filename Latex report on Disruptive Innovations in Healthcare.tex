\documentclass[12pt]{article}
\begin{document}
\title{A Latex report on Disruptive Innovations in Healthcare}
\maketitle{}

\section{Introduction}
Innovation is no stranger to the healthcare sector. New therapies, medical devices, and healthcare management practices are adopted all the time. However, up until fairly recently, examples of disruptive innovations in healthcare were far less common. What is disruptive innovation, and what impact do these disruptions have on the healthcare industry? Disruptive innovations are those that cause radical change and often result in new leaders in the field. They overturn the usual way of doing things to such an extent that they have a ripple effect throughout the industry. The following nine examples of disruptive innovations in healthcare are centered on technology, customer-centric care, and third-party advancements.

Technology is the biggest driver of many disruptive innovations in healthcare since every aspect of healthcare is dependent on some form of tech. From wearables and mobile phone apps to big data and artificial intelligence (AI) use in diagnosis, any new technology could potentially shake up healthcare.

\section{Different Innovations in Healthcare}
1. Consumer devices, wearables, and apps        
In the past, a patient could get only biometric data about their pulse, heart rate, blood oxygen, and blood pressure when they went to the doctor’s office. Now, consumers take charge of their own health journey, using data gathered from their Fitbits, smartwatches, and mobile phone fitness apps. Physicians can use the data gathered from these wearables to make treatment decisions, although the vast amount of personal information collected by these apps has led to legal and ethical concerns over data privacy.

2. AI and machine learning            
AI applications can manage patient intake and scheduling as well as billing. Chatbots answer patient questions. With natural language processing capabilities, AI can collate and analyze survey responses. AI will probably increase in use as a way to bring down healthcare costs and let doctors and staff focus on patient care. Healthcare leaders must be knowledgeable about the issues surrounding database management and patient privacy. 

3. Blockchain             
Blockchain is a database technology that uses encryption and other security measures to store data and link it in a way that enhances security and usability. This innovation facilitates many aspects of healthcare, including patient records, supply and distribution, and research. Tech startups have entered the healthcare sector with blockchain applications that have changed how providers use medical data. 

4. Electronic health records and big data              
Electronic health records (EHRs) have been a growing part of patient care since the adoption of the Affordable Care Act. The massive amount of EHR data goes far beyond patient health records, however, and can be used to conduct research, improve care, build AI applications, and create new business opportunities. Therefore, healthcare providers have to be aware of the issues surrounding EHR security.

5. Retail competition            
In 2019, Walmart formed Walmart Health, freestanding clinics that provide primary and urgent care. The same year, Amazon bought the online pharmacy PillPack, setting itself up to move into the pharmaceutical retail market and potentially disrupt the pharmacy benefits management market. In 2018, CVS acquired Aetna, moving from retail into health plans. All of these moves create new giants in the industry, changing the way healthcare operates. 


\section{Conclusion}
As these examples of disruptive innovations in healthcare show, health sector executives must be nimble and tech-focused to not only meet industry challenges but also turn them into opportunities for growth. They must also be hyper-aware of patient privacy and HIPAA regulations, as many of these innovations — wearables, EHRs, big data, and more — have a direct impact on those areas. 

Disruptive innovations in healthcare can influence a new system that provides a continuum of care focused on each individual patient's needs, rather than focusing primarily on complex disorders and urgent health crises. Because of advances in diagnostic and therapeutic technologies, NPs and physician assistants can competently diagnose and treat disorders that would have previously required a physician.

\end{document}